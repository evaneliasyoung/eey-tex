\documentclass[
    subject = {{Comp Sci}},
    course = {{1570}},
    section = {{101}},
    assignment = {{Assignment 10 \\ Final Project}},
    name = {{Student One ; Student Two}},
    email = {{stosid@mst.edu ; sttsid@mst.edu}}
]{eey-homework}

\usepackage{blindtext}

\begin{document}

\section{Fine Print}
\textbf{Due:} Wednesday, May 6, 2020, at noon, 100 points

For this assignment, you will submit multiple files containing a program written in C++.
You know how to use the submit program by now; I needn't tell you.
\textbf{
    Also, for this assignment, you are permitted to work with a partner.
    Yah, two of you can work together on this program.
    The only rule is that you both have to have the same instructor, Price or Brown.
    I'm not sure how you will do this given the pandemic situation we are in, but you may do so if you wish.
    It is not required, but we encourage it.
    The only other rule is that you both code - no spongers!!
}

\section{Background}
homework \#9.
This gives you the basic foundation upon which to build your program.
You have the school, janitor, and lunch classes.
You are going to add another class. It is the: \textbf{drink class:}
\begin{itemize}
    \item[--]{member variables}
    \begin{itemize}
        \item{a member string for description (whiskey or coffee)}
        \item{a float member for "effect". The value of this member will be determined for each object by the constructor. It will be interpreted this way: if the value is negative, it is coffee; if it is positive, it is whiskey. Zero is an unacceptable value. This value will affect its consumer by increasing or decreasing his (the janitor, of course) bac by that amount.}
    \end{itemize}
    \item[--]{member function}
    \begin{itemize}
        \item{A constructor that will assign to its "effect" variable at random from the following intervals: [-.010, -0.001] and [0.05, 0.10]. This will obviously determine the name of the drink, too.}
    \end{itemize}
\end{itemize}

\section{Details}
Thus, to start a simulation, you must declare a "clean" school (no stuff in it).
Then declare a janitor and place it in the center of the school.
Then declare a lunch and place at some random place in the school, and not on top of the janitor of course.
Likewise, scatter  randomly the required number of traps and drinks.
How you do all this is up to you.
You might even want to have a school constructor do a lot of this work.
Think about it.
Once everything is in place, get your actors (the lunch and the janitor) moving.\hfill\break

So what's this "config file" business? Well, you have had many constants in your programs in the past, and you will certainly have constants in this program.
But one of the reasons to have them is that they may be parameters to change as you run your program in different ways.
In order to change the parameters, though, you have to change the constants and then recompile your program.
This works, but there is another way.
For parameters that are going to change enough times that you are annoyed by recompiling, you can put those values in a "config.dat" file, just like a data file.
Then, for each different run of your program, you can just open the file and change one of the parameters, then run again.
You don't have to recompile.
This is not a big deal with little programs like ours, but some programs can take hours to compile!\hfill\break

This is the number of days (simulation calls), grid size, num drinks, num traps, window frequency.
Keep this order.
Do not put comments or anything else in the file.
Then, in your main function you will create a stream to connect to the config file and stream into local variables those parameters.
Cool, eh? (Yes, cool.)
All other constant-kinda-stuff should be constants in an appropriate header file. By the way, make the symbol for a lunch a constant 'L' and put it in a header.
That way you can code janitor to look for a specific constant symbol.

\section{Details Upon Details}
There is the issue of the drinks.
Clearly, you will have two kinds of drinks, coffee and whiskey.
Since the floor in the school is a grid of chars, you can't actually place different drink objects out there.
That would require really advanced c++ techniques - c+++++ concepts! So, you're going to do this in a novel way.
You see in the pseudo-code in the sim function that you declare an array of drinks of CONSTANT size.
You must do that, as you know, because the compiler must know the size of an array at compile time.
That \verb|MAX_DRINKS| can be set by you to something like 20.
But, you will only use a subset of them.
That value is the value read in from the config file: num drinks.
So, you will scatter at random that number of d's over the grid.
Then, whenever the janitor lands on a drink, he will choose the (num\_drink-1)th drink from the array of drinks to drink and be affected by, and then decrement num drinks.
In this way there is a 1-1 correspondence between the d's on the grid and the drinks in the array.
Also, you must understand this.
When you declare an array of user-defined objects (like drinks), the compiler will automatically call the default constructor for each of those objects in the array.
So, you need to build into your own default constructor the random assignment of "effect" member for that drink.
Finally, in my pseudo-code, I have the array of drinks outside the school class.
You might want to include the array of drinks as part of the school class.
This decision is up to you.\hfill\break

The janitor will start with a bac of 0.05.
At this level, he is considered sober.
He will move one step at a time with semi-intelligence.
That is, his step function will be passed the lunch object so that he can "see" where it is.
Knowing its location, he can move toward the quadrant it's located in.
Once his bac reaches "drunk" threshold (0.08), he reverts back to random steps.
If his bac reaches death level (0.25), he dies and that run and the simulation ends.
No matter how the janitor moves (randomly or not):

\begin{itemize}
    \item if he hits a wall, he doesn't move but does increment his bruise count.
    \item if he hits a window, he falls out.
    \item if he lands on drink, (he) removes the char from the array and "drinks" the drink.
    \item if he lands on a sandwich trap, he destroys it (thus you remove that char) and gets 5 bruises.
\end{itemize}

The lunch will move about randomly.
If, however, it lands on a cell with a sandwich trap, it has to remain there until the end of the day.
In that case, remove the symbol for trap and replace it with the L.
You might want to include in your lunch class a member variable to indicate whether or not it is trapped.
This and other design decisions are yours to make.
If the lunch hits anything else (including the janitor! (stupid lunch)) it doesn't move anywhere that turn.

\section{Optional}
Now, after you get all this stuff done, you have the option to add "your own".
That's right, if you can think of fun stuff to add to this, do so.
You will be judged on how much you make your instructor laugh.
If you really impress them with something innovative and goofy, then maybe they'll give you points for it.
Good luck!!\hfill\break
\textbf{When you submit, submit under only one name} and be sure to include in your comment header the names of your partners on this project.

\section{Addendum}
The code you submit should output to the screen the first simulation (first day of chasing a sandwich) ONLY.
That is, after each move by the janitor or the sandwich, output to the screen the 2-D array representing the school (output the school).
If you make no provision to "slow this down", then all the movement by the actors will go by so fast it will be imperceptible.
You can make this appear on the screen so that you can watch the motion by including a few simple lines of code.
After your output statement, put this line of code: \verb|usleep(200000);| This pauses the processing for a fifth of a second.
You will have to \verb|#include <unistd.h>| for this to work.
Try this now with what you have.
Make a for loop that has the janitor step randomly 25 times and put this code in and you can watch him walk about.
And remember, do this ONLY for the first day; the rest of the simulations should NOT be output.

\end{document}
